\chapter{Einrichtung}
	\section{Konfiguration}\label{konfiguration}
		In der Konfigurationsdatei \texttt{config/app.xml} können grundlegende
		Einstellungen vorgenommen werden. Diese Datei ist vollständig dokumentiert und
		prinzipiell selbsterklärend.
		
		Die wahrscheinlich wichtigste Änderung ist die des Titels, welcher über
		\texttt{<app:name>} festgelegt werden kann.
		
		Weiterhin können sowohl der Server, mit dem sich verbunden werden soll, sowie
		die Datenbank, die angezeigt werden soll, hier geändert werden.
		
		Schlussendlich können die gewünschten Anmeldeverfahren (Windows-Anmeldung,
		SQL-Server-Anmeldung) und das Farbschemata für das Programm eingestellt
		werden.
	
	\section{Installation}
		Es wird vorausgesetzt, dass der Datenbankserver bereits eingerichtet wurde.
		Dies schließt die Einrichtung eines entsprechenden Datenbankschemas und eines
		ersten Benutzers mit Zugriff auf die Datenbank ein. Sofern der Name des
		Datenbankschemas vom Standardnamen \texttt{autohaendler} abweicht, muss dies
		in der Konfigurationsdatei geändert werden.
		
		Um sämtliche Tabellen zu erzeugen, kann je nach Anmeldeart einer der folgenden
		Befehle genutzt werden.
		
		\subsection*{Mittels Windows-Anmeldung}
			\begin{verbatim}
				java -cp autohaendler.jar net.mabako.migratetool.MigrateTool [server]
			\end{verbatim}
			
			Der Parameter \texttt{server} ist optional und wird alternativ aus der
			Konfigurationsdatei geladen, standardmäßig ist \texttt{localhost}
			voreingestellt.
			
		\subsection*{Mittels SQL-Server-Anmeldung}
			\begin{verbatim}
				java -cp autohaendler.jar net.mabako.migratetool.MigrateTool [server]
				[benutzername] [passwort]
			\end{verbatim}
	
			Alle Parameter werden benötigt.

\chapter{Anwendung}
	\section{Anmeldung}
		Es besteht die Möglichkeit, sich über die Windows-Authentifizierung anzumelden
		oder einen Benutzernamen und das dazugehörige Passwort einzugeben.
		Nach der Anmeldung wird die Hauptansicht angezeigt, in der jede Tabelle der
		Datenbank, als Button repräsentiert, aufgeführt ist.
		
		Die Konfiguration einzelner Anmeldemethoden erfolgt über die \texttt{app.xml},
		wie in Kapitel \ref{konfiguration} beschrieben.
		
	\section{Tabellenwahl}
		Durch einen linken Mausklick auf einen der Tabellennamen wird diese geöffnet.
		Über den ``Zurück'' Button in der linken oberen Ecke wird wieder in die
		Hauptansicht gewechselt.
		
	\section{Datensatz eingeben und ändern}
		Datensätze können über Tastatur eingeben werden. Zu beachten ist, dass
		Dezimalzahlen nicht mit einem ',' sondern einem '.' eingegeben werden müssen.
		
		Um Datensätze zu ändern, müssen vorher die entsprechenden Einträge mit der
		Rücktaste gelöscht werden, da der Cursor immer automatisch am Ende der
		Zeichenkette steht.
		
		Das Feld \texttt{id} wird automatisch gefüllt, nachdem der Datensatz
		alle benötigten Felder ausgefüllt wurden und erfolgreich angelegt wurde.
		Sofern in der Spalte ein Wert für \texttt{id} vorhanden ist, wird der
		Datensatz nach jeder Änderung gespeichert.
		
	\section{Datensatz löschen}
		Um Datensätze zu löschen, müssen die entsprechenden Zeilen, unter Umständen
		mithilfe der Shift- und Strg-Tasten ausgewählt werden, bevor der "`Ausgewählte
		löschen"'-Button betätigt wird.
		
	\section{Tabelle drucken}
		Jede Tabelle kann über den Button "`Drucken"' ausgegeben werden.
		Daraufhin erscheint das übliche systeminterne Druckmenü, in dem Drucker
		ausgewählt und Seiteneinstellungen vorgenommen werden können.
	
	\section{Bestellungen der einzelnen Kundenanzeigen}\label{bestellungen_anzeigen}
		Bei Verwendung der Kunden-Tabelle können
		für jeden Kunden neue Bestellungen anzeigt und existierende verändert werden.
		Dazu muss genau ein Kunde ausgewählt werden und danach die Schaltfläche
		"`Bestellungen anzeigen"' verwendet werden.
	
	\section{Einzelne Posten in einer Bestellung anzeigen}
		Analog zum Anzeigen der Bestellungen eines Kunden können die Autos in dieser
		Bestellung angezeigt werden, dazu kann die Schaltfläche "`Autos anzeigen"'
		genutzt werden.
		
		Der Gesamtbetrag der Rechnung wird bei Hinzufügen, Ändern und Löschen von
		Posten in der Bestellung aktualisiert, jedoch muss die Tabelle für die
		korrekte Anzeige noch einmal neu geladen werden. Dies kann über Auswahl von
		"`Zurück"' und nachfolgender Auswahl der Bestellungen der Kunden
		(Kapitel \ref{bestellungen_anzeigen}) bzw. der gesamten Liste der Bestellungen
		(im Hauptmenü) erfolgen.

