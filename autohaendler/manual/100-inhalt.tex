\chapter{Einrichtung}
	\section{Konfiguration}
		In der Konfigurationsdatei \textbf{config/app.xml} können grundlegende
		Einstellungen vorgenommen werden. Diese Datei ist vollständig dokumentiert und
		prinzipiell selbsterklärend.
		
		Die wahrscheinlich wichtigste Änderung ist die des Titels, welcher über
		\texttt{<app:name>} festgelegt werden kann.
		
		Weiterhin können sowohl der Server, mit dem sich verbunden werden soll, sowie
		die Datenbank, die angezeigt werden soll, hier geändert werden.
		
		Schlussendlich können die gewünschten Anmeldeverfahren (Windows-Anmeldung,
		SQL-Server-Anmeldung) und das Farbschemata für das Programm eingestellt
		werden.
	
	\section{Installation}
		Es wird vorausgesetzt, dass der Datenbankserver bereits eingerichtet wurde.
		Dies schließt die Einrichtung eines entsprechenden Datenbankschemas und eines
		ersten Benutzers mit Zugriff auf die Datenbank ein.
		
		Um sämtliche Tabellen zu erzeugen, kann je nach Anmeldeart einer der folgenden
		Befehle genutzt werden.
		
		\subsection*{Mittels Windows-Anmeldung}
			Der Parameter \texttt{server} ist optional und wird alternativ aus der
			Konfigurationsdatei geladen.
			
			\begin{verbatim}
				java -cp autohaendler.jar net.mabako.migratetool.MigrateTool [server]
			\end{verbatim}
		
		\subsection*{Mittels SQL-Server-Anmeldung}
			Alle Parameter werden benötigt.
			
			\begin{verbatim}
				java -cp autohaendler.jar net.mabako.migratetool.MigrateTool [server]
				[benutzername] [passwort]
			\end{verbatim}
	
\chapter{Anwendung}
	\section{Anmeldung}
		Es besteht die Möglichkeit, sich über die Windows-Authentifizierung anzumelden
		oder einen Benutzernamen und das dazugehörige Passwort einzugeben.
		Nach der Anmeldung wird die Hauptansicht angezeigt, in der jede Tabelle der
		Datenbank, als Button repräsentiert, aufgeführt ist.
		
	\section{Tabellenwahl}
		Durch einen linken Mausklick auf einen der Tabellennamen wird diese geöffnet.
		Über den ``Zurück'' Button in der linken oberen Ecke wird wieder in die
		Hauptansicht gewechselt.
		
	\section{Datensatz eingeben}
		Datensätze können über Tastatur eingeben werden. Dabei wird die ID für
		neue Datensätze immer automatisch erstellt und muss nicht extra angeben
		werden.
		Zu beachten ist, dass Dezimalzahlen nicht mit einem ',' sondern einem '.'
		eingegeben werden müssen.
		
	\section{Datensatz ändern}
		Um Datensätze zu ändern, müssen vorher die entsprechenden Einträge mit der
		Rücktaste gelöscht werden, da der Cursor immer automatisch am Ende der
		Zeichenkette steht.
		
	\section{Datensatz löschen}
		Um Datensätze zu löschen, müssen die entsprechenden Zeilen mit Hilfe der
		Shift- und Strg-Tasten ausgewählt werden, bevor der ``Ausgewählte löschen''
		Button betätigt wird.
		
	\section{Tabelle drucken}
		Zum Drucken einer Tabelle muss der ``Drucken'' Button betätigt werden.
		Daraufhin erscheint das übliche Druckmenü, in dem Drucker ausgewählt und
		Seiteneinstellungen vorgenommen werden können.
	